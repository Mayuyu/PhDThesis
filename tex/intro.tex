% !TeX root = ../thesis.tex

%# -*- coding: utf-8-unix -*-
%%==================================================
%% chapter01.tex for SJTU Master Thesis
%%==================================================

%\bibliographystyle{sjtu2}%[此处用于每章都生产参考文献]
\chapter{带多尺度和不确定性的输运与波动问题的计算方法:简介}
\label{chap:intro}

本文中,我们主要关注于一大类常见的波动方程和Kinetic方程,包括对流方程和刘维尔方程、输运方程、薛定谔方程及分数阶的守恒律方程。我们主要的目的是相关的计算方法的设计与分析。这些方程通常来源于物理问题或工程问题,在实际应用中具有重要的意义,在数学尤其是应用数学的研究中又具有非常强的代表性。如对于对流方程、刘维尔方程及输运方程,相关的经典计算方法的设计和分析虽然相对成熟,但是在实际的物理问题或者工程问题中,由于各种误差和不确定性的存在使得这些经典方法的应用大大受限,这时,如何对这种误差或不确定性带来的影响进行数值估计和分析就变得非常重要了,这也正是我们在本文中讨论的重点之一(第二章和第三章)。而对于薛定谔方程,由于属于所谓的高频波问题,会使通常的数值格式的计算效率非常低下甚至不可用,这类问题目前还没有特别完美的数值格式可以在实际中大规模使用,针对这一点,本文试图通过改进一类特定的薛定谔方程的快速算法来做出一些自己的贡献(第四章)。最后,对于分数阶的守恒律方程,属于近年来新兴的一类问题,特别是在物理中的所谓多孔介质中的记忆效应的研究中非常重要,并且这类问题的已有研究相对较少,本文中通过构造相应的数值格式,期望增加我们对相关现象的理解(第五章)。

在本章中,我们将对相关问题作简要的介绍,主要包括研究的背景,研究的动机以及研究的内容和成果。
\section{物理方程的不确定性量化}

数值模拟的终极目标是预测物理事件或工程系统的行为。人们投入了大量的努力来开发精确的数值算法,目的是使得在数值误差得到很好的控制和理解的意义上,模拟预测是可靠的。这是数值分析的主要目标,并且仍然保持着很高的活跃度。然而,在传统意义的数值分析中,对于参数值、初始和边界条件等“数据”中错误或不确定性的影响的很少关心。不确定性量化(Uncertainty Quantification,简称UQ)的目标是研究这些错误对数据的影响,并随后为实际问题提供更可靠的预测。这个领域在过去几年中受到越来越多的关注,特别是在复杂系统的背景下,数学模型只能作为真实物理学的简化和缩小的表示。虽然许多模型已经成功地揭示了预测和观测之间的量化关系,但是它们的使用受到我们在控制方程中为各种参数分配精确数值的能力的限制。由于我们对潜在的物理规律或不可避免的测量误差了解的并不完整,不确定性正代表了数据的这种变异性并且时刻存在。因此,为了充分了解模拟结果和随后的真实物理学,必须从模拟一开始就引入不确定性,而不是之后。

很多人在土木工程,水利,控制论等学科领域长期以来已经认识到了解不确定性的重要性。随之而来的是解决这个问题的方法。 由于不确定性的“不确定”的性质,最主要的方法是将数据不确定性视为随机变量或随机过程,并将原始确定性系统重写为随机系统。
我们指出,这种类型的随机系统与经典的“随机微分方程”(SDE)不同,其中随机输入是一些理想化的过程,如维纳过程,泊松过程等,并且诸如随机分析的工具已经被广泛发展,仍在积极研究之中,参见例如\citen{Gard1985,KP1999,KS1988,O1998}。
\subsection{已有方法简介}

\subsubsection{蒙特卡罗和基于抽样的方法}

最常用的方法之一是蒙特卡罗取样(Monte Carlo Sampling,MCS)或其变体之一。在MCS中,根据预先给定的概率分布,生成随机输入的一些(独立)实例。对于每个实例,数据是固定的,问题变成确定。在解决问题的确定性实例之后,人们收集一组解决方案,即随机解的实例。从这个集合可以提取统计信息,例如均值,方差等。尽管MCS的应用是非常直接的,因为它只需要重复执行确定性的模拟,通常需要大量的重复,解的统计量的计算相对较慢。例如,平均值通常收敛为$\dfrac{1}{\sqrt K}$,其中K是实例的数量(例如\citen{Fish1996})。对于准确结果的大量实现的需要可能导致过多的计算负担,特别是对于在其确定性设置中已经具有计算密集度的系统。
人们已经开发了用于暴力加速MCS的收敛的技术,例如拉丁超立方体采样(参见\citen{loh1996latin,stein1987large}),准蒙特卡罗(参见\citen{fox1999strategies,niederreiter1992random,niederreiter2012monte})等等。其中值得一提的进年来发展迅速、具有很大潜力的多级蒙特卡罗方法(Multilevel Monte Carlo,MLMC),这种方法一种递归控制变量策略,使用一些随机输出量的廉价不准确的近似作为控制变量,以获得更准确但成本更高随机量的近似值,详细的介绍可以参考文献\citen{giles2015multilevel}。
然而,这些方法的设计同常有很多额外的限制,并且它们的适用性通常是有限的。

\subsubsection{扰动方法}

最流行的非抽样方法是扰动方法,其中随机场通过泰勒级数在平均值附近展开,并在某些项截断。通常,最多使用二阶展开,因为高阶展开所得到的等式的结果变得非常复杂。这种方法已被广泛应用于各种工程领域\citen{liu1986probabilistic,liu1986random}。 扰动方法的固有局限性在于,输入和输出的不确定性的大小不能太大(通常小于10%),而且这些方法在其他方面表现不佳。

\subsubsection{矩方程方法}

在这种方法中,人们试图直接计算随机解的矩。未知数是解的各阶矩,其方程是通过取原始随机控制方程的平均得出的。 例如,平均场由控制方程的平均值确定。困难在于,矩方程的导出,除了在极少数情况下,总是需要更高阶矩的信息。 这就引出了所谓的“矩封闭”问题,这通常是通过利用一些关于更高阶矩的特别性质来解决的。在水利学背景下,矩方程的更详细的介绍可以在\citen{zhang2001stochastic}中找到。

\subsubsection{基于算子的方法}

这些方法是基于控制方程中随机算子的运算。它们包括Neumann展开,其表示Neumann级数中的随机算子的逆\citen{shinozuka1988response,yamazaki1988neumann}和加权积分法\citen{deodatis1991weighted1,deodatis1991weighted2}。类似于扰动方法,这些基于运算子的方法也被限制在很小的不确定性的情况下。 它们的适用性通常强烈依赖于底层操作员,并且通常限于静态问题。

\subsection{广义多项式混沌(gPC)}

最近发展的,广义多项式混沌(generalized Polynomial Chaos, gPC)\citen{XiuKar,GS},即经典多项式混沌的推广,已经成为最广泛使用的方法之一。使用gPC,随机解被表示为输入随机参数的正交多项式,并且可以选择不同类型的正交多项式以实现更好的收敛。 它本质上是随机空间中的谱表示,并且当解光滑地依赖于随机参数时,其表现出快速收敛。基于gPC的方法将成为本文的重点。

\subsection{gPC方法的发展与应用}

gPC的发展开始于R.Ghanem及其同事们在PC(多项式混沌)上的创新工作。受Wiener-Hermite均匀混沌理论的启发(Ghanem)\citen{Wie}),Ghanem采用Hermite多项式作为正交基来表示随机过程,并将该技术应用于许多工程问题的获得了成功。概述可以在\citen{GS}中找到。

Hermite多项式的使用虽然在数学上是完备的,但在一些应用中,特别是在非高斯问题的收敛和概率近似的情况下,存在困难\citen{chorin1974gaussian,orszag1967dynamical}。随后,在\citen{XiuKar}中提出了广义多项式混沌(gPC)来减轻困难。在gPC中,根据随机输入的概率分布,选择不同种类的正交多项式作为基。可以通过选择适当的基来实现最优收敛。在一系列论文中,gPC的强大在各种PDE中得到了证明\citen{xiu2002modeling,xiu2003modeling}。

对于gPC的工作进一步推广是不需要作为基的多项式全局光滑。原则上任何一组完备的基都是可行的选择,就像有限元法一样,取决于给定的问题。这种推广包括分段多项式基\citen{babuska2004galerkin},小波基\citen{le2004multi}和多元gPC\citen{wan2005adaptive}。

\subsection{侵入性方法(Intrusive Method): 随机伽辽金方法}

当应用于具有随机输入的微分方程时,要求解的量是gPC展开的系数。典型的方法是进行伽辽金投影以最小化有限阶gPC展开的误差,并且因为用于扩展系数的求解方程组是确定性的,可以通过常规数值方法来解决。这种随机伽辽金(stochastic Galerkin,SG)方法,已经在PC的早期工作中应用,并被证明是有效的,也是本文中主要使用的方法。

\subsection{非侵入性的方法(Non-intrusive Method):随机配点法}

近年来,随着\citen{xiu2005high}的工作,高阶随机配点(stochastic collocation,SC)方法的兴趣激增。这在某种程度上是对旧技术“确定性抽样方法”的重新发现。随机配点方法的早期工作包括\citen{mathelin2003stochastic,tatang1997efficient},并使用一维正交点的张量积为“抽样点”。虽然已经表明这种方法可以实现高阶逼近,但参见\citen{babuvska2007stochastic},其适用性仅限于较少数量的随机变量,因为采样点的数量以指数速度增长。\citen{xiu2005high}的工作从多元插值分析中引入了“稀疏网格”技术,可以显着减少较高随机维数的采样点数。SC算法的实现与MCS类似,即只需要通过选择适当的采样点集合,使用确定性求解器进行重复实现。原始高阶随机配点中,基函数是由节点定义的拉格朗日多项式,即稀疏网格\citen{xiu2005high}或张量网格\citen{babuvska2007stochastic}。\citen{xiu2007efficient}提出了一种更实用的“伪谱”方法,可以在gPC逼近的基础上与配点法进行结合。伪光谱gPC方法在实践中被证明比拉格朗日插值法更容易实现。

\subsection{gPC方法小结}

由于gPC-SG方法在数值分析和实践中具有相当的简洁性,所以本文中我们将以gPC-SG方法为主,而使用gPC-SC方法作为参考对比。关于两者之间的优缺点及相关分析研究,读者可以参考\citen{xiu2009fast}。


\section{本文的主要内容与创新点}


\subsection{带有随机、间断系数的双曲型方程方程:提出离散gPC-SG方法}

这类问题出现在异质介质中的波传播中,通过不同介质之间的界面或潜在的障碍,使这些方程中的系数产生间断。随机或不确定性来自建模或实验误差。

为了处理间断系数,我们需要在间断点提供额外的物理条件,并在哈密顿保持格式的框架内\citen{Jin:2009pro, Qi:2013byba, Wen:2005ueba, JinWen-wave},将这种条件构建到数值通量中。为了处理不确定性,我们使用gPC-SG方法。标准gPC-SG方法从原始方程的gPC近似开始,得到gPC系数的确定性方程组,然后通过空间和时间的标准格式离散化。这里的主要困难在于gPC近似仅当原始问题的解在随机空间中是光滑时才是准确的,而在我们的问题中解却具有高度奇异性。

我们提出了克服这个困难的新方法,主要想法是逆转上述gPC-SG过程。也就是说,我们首先用空间和时间离散原始方程,使用光滑的数值通量,然后将gPC近似应用于该离散方程。由于离散的解比连续的解更光滑,所以gPC近似被应用于更光滑滑的函数(对于固定时间步长和网格大小),因此可以得到很好的收敛速度。

\subsection{带有不确定性的输运方程:gPC-SG方法的一致(关于克努森数)收敛性分析与micro-macro格式的构建}

在这个问题上,我们并没有遇到不光滑的解,但这里的问题是不确定性和多尺度并存。我们研究了在碰撞横截面,初始数据或边界数据中包含不确定性的线性输运方程。多尺度性质的特征由克努森数号$\varepsilon$表示,其在所谓的光学薄区域$(\varepsilon \ll 1)$中,由于粒子的高散射率,导致线性传输方程趋近到扩散方程,称为扩散极限。近来,开发具有不确定性和扩散尺度的线性传输方程的渐近保持(AP)格式非常活跃(在随机Galerkin方法的框架下,成为s-AP方法)\cite{JXZ}。如果当$ \varepsilon \rightarrow 0 $时,用于线性传输方程的随机Galerkin方法收敛到用于扩散极限方程的随机Galerkin方法,则方法是s-AP的。

这个研究的主要困难在于,当$\varepsilon \ll 1 $时,能量估计和收敛速度通常取决于$\varepsilon $的倒数,这意味着需要使用的gPC近似多项式的阶数随着$\varepsilon $减少而剧烈增加。虽然AP格式可以使用独立于$\varepsilon $的数值参数,但是要严格证明这一点非常困难。

在本研究中,我们为随机碰撞横截面的线性传输方程提供了随机Galerkin方法的最佳误差估计,这是首次由人得到关于这类问题的一致收敛结果。对于数值部分,我们使用基于micro-macro分解的方法开发完全离散的s-AP方法。这种方法的优点在于它允许我们的到不依赖于$\varepsilon$的稳定性条件。

\subsection{使用非均匀快速傅立叶变换(NUFFT)改进具有向量势的半经典薛定谔方程的快速算法}
% This research following the previous work done Jin and Zhou~\cite{Jin_Zhou}. In order to design an unconditional stable scheme, a semi-Lagrangian time splitting method was introduced where the meshing strategy $\Delta t=O(\varepsilon)$ and $\Delta x=O(\varepsilon)$ is sufficient to guarantee an accurate approximation of the wave function. here the small parameter $\varepsilon$ stands for the scaled Planck constant. Similarly, one can use $\varepsilon$ independent time steps to capture correct physical observables. In the convection step, a polynomial interpolation technique was analyzed and implemented in~\cite{Jin_Zhou}, where the spatial accuracy was sacrificed for efficiency consideration. In fact, a spectral interpolation can be applied instead to improve spatial accuracy, unfortunately, it would increase the computational complexity from $O(N)$ (polynomial interpolation) to $O(N^2)$ (direct Fourier series summation), where N is the number of grid points. The primary issue is that the standard inverse FFT no longer applies since the evaluation points are not necessarily uniformly spaced.
% Hence, a balance between efficiency and accuracy is desired for the semi-Lagrangian method.

% Thanks to the nonuniform Fast Fourier transform (NUFFT) (see~\cite{Dutt, greengard}), the problem can be solved ideally. This is the major motivation of our work. we incorporate the NUFFT algorithm into the time splitting semi-Lagrangian method, and the computational complexity is $O(N\log N)$. The new method is proved to be unconditionally stable. Unlike the polynomial case, where special care is needed for the interpolation stencil, the interpolation is now done by a global spectral approximation. When time and space oscillations are resolved, namely, $\Delta x = O(\varepsilon)$ and $\Delta t = O(\varepsilon)$, we prove that our method is spectrally accurate in space and first order accurate in time. We also showed, in the framework of Wigner transform, that $\varepsilon$-independent time
% steps are allowed to compute the correct physical observables.

为了设计解带有向量势的电磁场中的薛定谔方程的无条件稳定的格式,Jin和Zhou在\citen{SemiJZ}中引入了一种半拉格朗日时间算子分裂方法,其中网格划分策略$\Delta t=O(\varepsilon)$和$\Delta x=O(\varepsilon)$就足以保证波函数的精确近似。这里的小参数$\varepsilon $代表缩放的普朗克常数。类似地,可以使用与$\varepsilon $独立的时间步长来捕获正确的物理可观察量。在对流骤中,多项式插值技术在\citen{SemiJZ}中进行了分析和实现,为了效率考虑牺牲了空间精度。实际上,如果使用谱插值就可以提高空间精度,不幸的是,它会将计算复杂度从$O(N)$(多项式插值)增加到$O(N^2)$(直接傅里叶级数求和)其中$N$是网格点的数量。这里的主要问题是标准逆FFT不再适用,因为取样的点不一定均匀分布。

由于非均匀快速傅里叶变换(NUFFT)(见\citen{nufft2,nufft6})的提出,问题可以理想地解决。这是我们工作的主要动力。我们将NUFFT算法合并到时间分裂半拉格朗日方法中,计算复杂度为$O(N\log N)$。我们证明了该方法是无条件稳定的。与多项式插值的情况不同,现在通过全局谱近似来进行插值。当需要还原时间和空间振荡时,即$\Delta x = O(\varepsilon)$及$\Delta t = O(\varepsilon)$,我们证明我们的方法在空间和时间上的额准确的。我们还在Wigner变换的框架中证明,使用与$\varepsilon $独立的时间步长即可允许我们计算正确的物理观测值。

\subsection{具有分数阶导数(Caputo)的守恒律方程的数值分析与计算}

这项研究属于经典数值分析和计算,问题不具有不确定性。但是与经典守恒律不同,对于时间导数,我们使用Caputo导数($\partial_t^\alpha$, $\alpha\in (0,1]$),其引入了多孔流体扩散与记忆这种非局部记忆效应,以及与非线性守恒定律相结合,我们对数值分析和计算两个方面的兴趣。

我们提出了一阶和二阶显示格式,并显示了使用修改的CFL条件,数值方案是TVD(总变差减小)的。然而,修改的CFL条件越来越受限于$\alpha\rightarrow 0$,这使得显式格式对于小的$\alpha$不可行。受此影响,我们进一步设计了一种隐式的迎风格式,该方法被证明是无条件地稳定的,并且同样也是TVD的。在这些格式的帮助下,数值计算证明了方法的有效性,并提供数值证据来解释了守恒律中的记忆效应。据我们所知,这是对分数导数的非线性守恒定律的少数尝试之一。





