%# -*- coding: utf-8-unix -*-
% !TeX root = ../thesis.tex

\begin{thanks}

在美丽安静的交大校园里,我度过了生命中无比珍贵的几载年华,完成了论文和学业,增添了知识和能力,更收获了宝贵的人生财富!回想起一路走过的既艰辛又快乐的历程,几载的成长和进步若仅凭一己之力是不可想象的,感激之情不禁油然而生。

首先要感谢的是我的博士导师金石教授。论文的顺利完成离不开金老师的悉心指导。从论文的选题、开题、写作到最后定稿的各个环节,无不倾注了导师的大量心血。几年来,金老师把我从一个稚嫩的本科生,引入计算数学的大门,逐步引领到科研的最前沿。多次资助我到国外顶尖学府交流访问、参加相关领域顶级的学术会议,使我得到了远多于同龄人的机会。在这过程中,大大地开拓了我的科研视野,获得了和相关领域顶级大师交流的机会,为我今后的科研铺平了道路。从导师那里,我不仅得到科学研究的系统训练,而且还亲身领略到了导师的大家风范,特别是导师高尚的人格、渊博的学识、严谨的治学态度、敏锐的洞察力、不懈探索的精神、诲人不倦的师德,以及谦和的为人处事方式,都令我受益良多,成为我一生受之不尽的宝贵财富。同时我要感谢美国杜克大学的刘建国教授,在金老师的引荐下,我有机会能够与刘老师合作,在刘老师的指导下完成博士论文中的部分内容。在这个过程中,刘老师也教会了我许多东西,他对数学的敏锐洞察力,对于数学的忘我的精神,使我认识到了自身的巨大差距,也使我有了努力的方向。可以说,能遇到金老师和刘老师并成为他们的学生是我一生中最大的幸运。

其次我要感谢交大致远学院和自然科学院的创建者蔡申瓯教授、鄂维南教授以及我的导师金石教授。他们作为国际上相关领域的顶尖学者,心系祖国,非常关心国内科研人才的培养。他们不辞辛苦,经常往返于中美之间,为了能给学生上课、指导学生科研以及学院的相关事务安排。他们还吸引了一系列国内外的知名学者加盟,像季向东教授、王立河教授、何小刚教授等等,本科期间他们均亲力亲为给我们上课,使我在人生中最好的年华能够享受到最好的教育。与此同时,交大数学系和物理系的老师们也对我关怀备至,像王亚光教授、黄建国教授、章璞教授等等,他们讲课风趣幽默、通俗易懂而细致入微,使我们打下了坚实的数学基础。同时我也要感谢在我读博士期间,自然科学研究院和
数学系的一批优秀的青年教授,像唐敏特别研究员、应文俊特别研究员、胡丹教授、周栋焯教授、徐振礼教授、张镭特别研究员、李敬来特别研究员、张小群特别研究员、魏星特别研究员等等,他们或者组织我们参加会议和讨论班,或者给我们上相关的课程,平时与我们交流讨论。这些青年教授活力四射,亲切可近而又学识渊博,我从中受益良多,他们就是我最好的榜样、努力的目标。

我还要感谢我的几位师兄师姐:周珍楠、柴利慧、李沁等等,他们有的带领我进行科研,同时以自己的亲身经验引导我,有的耐心的回答我的各种学术和生活上的问题。我也要感谢我博士的同学李新春、黄健超、刘沛、许志钦、张耀宇、肖彦洋、蒋诗晓、杜涛等等,有这样一批优秀的同学我感到非常骄傲,与他们讨论总能使我学到新的东西。同时还有本科的一些同学,虽然我们没在一起读博士,可能方向也各不相同,但是仍然总与我讨论科学上的问题,像邰骋、李赫、李志超谈安迪、马璟琛、赵清宇、赵浩然等等。与他们讨论,我获得了更广阔的视野,了解了各行各业的动态和最前沿。

最后,谨以此文献给我挚爱的双亲,他们在背后的默默支持一直是我前进的动力。从小时候的习惯的培养、兴趣的培养、特长的培养,他们是我最好的启蒙老师。在此祝愿年过六旬的他们身体健康,心情愉快!

\end{thanks}
