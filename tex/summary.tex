% !TeX root = ../thesis.tex

%# -*- coding: utf-8-unix -*-
%%==================================================
%% conclusion.tex for SJTUThesis
%% Encoding: UTF-8
%%==================================================

\begin{summary}

在本文中,我们讨论了对于一些物理中的波动方程与输运方程相关问题的数值算法与分析。主要包括带有不确定性的(随机性)的和经典的两大类问题。

对于不确定量化,首先,我们研究了带有间断与随机系数的双曲型方程的数值解法。为了克服由解的不光滑导致的gPC-SG方法收敛速度很慢的问题,我们提出了离散gPC-SG方法,利用离散的解具有较好的光滑性,进而改进gPC-SG方法的收敛速度。对于对流方程我们进行了收敛性的分析与误差估计。同时为了说明方法的有效性,我们进行了大量的数值实验,包括对流方程与刘维尔方程;对于线性输运方程,由于多尺度与不确定性的同时存在,我们建立了gPC-SG方法对带有随机散射系数的线性输运方程关于克努森数一致的谱精度分析,从而允许我们证明该方法随机渐近保持性质(s-AP)。对于基于micro-macro分解的全离散格式,我们证明了一致的稳定性结果。这是首次有人证明了关于这类问题的一致性结果。


对于传统的算法的改进,我们在第四章中,提出并分析了具有向量势的半经典薛定谔方程的新的时间分裂谱方法,其中在对流部分的半拉格朗日方法的插值步骤中应用NUFFT技术。分析了近似波函数和计算物理观测值的方法的稳定性和准确性。在最后,我们对近年来刚刚兴起的分数阶导数的波动方程进行了数值研究和分析,提出了相应的一阶、二阶显式、隐式迎风格式,对于稳定性和TVD特性进行了分析。我们通过大量的数值实验,包括Burgers方程等来说明我们的格式的可行性。借助于这类格式,我们得以通过数值实验来帮助我们理解所谓的记忆效应。

所有以上的问题中我们都进行了严格的分析论证,同时进行了大量的数值实验来说明我们方法的优越性。但是,还有很多没有解决的问题,例如离散gPC-SG方法在高阶格式、非线性问题的构造上仍然存在一定困难,同时如何将该方法向更广泛的问题上进行推广仍然需要大量的研究;关于一致性的分析在带有多尺度与不确定性的问题非常重要,我们的分析对于线性问题具有非常一般的指导作用,但对于非线性的问题,仍然非常困难;不确定量化中gPC-SG方法如何保持原方程的双曲性、一致谱收敛的结果能否向更多类型的方程推广;对于带有分数阶的PDE无论在分析还是在计算方法的研究上,都非常有价值,然而相关的数值分析与理论分析结果都非常有限,如何更好的理解记忆效应及相关的数学理论等等;这些将作为我以后的主要研究方向。


\end{summary}
