%# -*- coding: utf-8-unix -*-
\chapter{质量和能量守恒的证明}
\label{app_nufft}

% In this appendix, we briefly present the proof for mass and energy conservation.  We assume the wave function lies in 
% Schwartz space $S(\mathbb R^d)$ hereafter.
在本附录中,我们简要介绍了第四章中薛定谔方程的质量和能量守恒的证明。我们假设波函数在
Schwartz空间$S(\mathbb R^d)$中。

\section{质量守恒}
\begin{proof}
对于$u^{\varepsilon}\in C(\mathbb{R}_t;L^2(\mathbb{R}^d))$,首先注意到$(-i\varepsilon\nabla-\mathbf{A})$是自共轭算子,即,
\begin{equation}
\langle(-i\varepsilon\nabla-\mathbf{A})f,g\rangle=-i\varepsilon\langle\nabla f,g\rangle-\langle\mathbf{A}f,g\rangle
=\langle f,(-i\varepsilon\nabla-\mathbf{A})g\rangle.
\end{equation}
直接计算得出
\begin{equation}\nonumber
\frac{d}{dt}\| u^{\varepsilon}\|_{L^2}=\frac{d}{dt}\langle u^{\varepsilon},u^{\varepsilon}\rangle=\langle\partial_t u^{\varepsilon},u^{\varepsilon}\rangle+\langle u^{\varepsilon},\partial_t u^{\varepsilon}\rangle =0.
\end{equation}
\end{proof}
\vspace{-1.0cm}
\section{能量守恒}
\begin{proof}
对$\mathcal E(t)$求导,我们有 
\[
\frac{d}{dt}\mathcal E(t)=(I)+(II),
\]
其中
\begin{align*}
(I)&:=\frac{1}{2}\langle(-i\varepsilon\nabla-\mathbf{A})\partial_t u^{\varepsilon},(-i\varepsilon\nabla-\mathbf{A})u^{\varepsilon}\rangle+\frac{1}{2}\langle (-i\varepsilon\nabla-\mathbf{A})u^{\varepsilon},(-i\varepsilon\nabla-\mathbf{A})\partial_t u^{\varepsilon}\rangle, \\
(II)&:=\langle\partial_t u^{\varepsilon},Vu^{\varepsilon}\rangle+\langle u^{\varepsilon},V\partial_t u^{\varepsilon}\rangle.
\end{align*}
由$(-i\varepsilon\nabla-\mathbf{A})$是自共轭算子,我们有
\begin{align*}
(I)=&\frac{1}{2}\langle(-i\varepsilon\nabla-\mathbf{A})\left(\frac{1}{2i\varepsilon}(-i\varepsilon\nabla-\mathbf{A})^2 u^{\varepsilon}+\frac{V}{i\varepsilon}u^{\varepsilon}\right),(-i\varepsilon\nabla-\mathbf{A})u^{\varepsilon}\rangle\\
&+\frac{1}{2}\langle(-i\varepsilon\nabla-\mathbf{A})u^{\varepsilon},(-i\varepsilon\nabla-\mathbf{A})\left(\frac{1}{2i\varepsilon}(-i\varepsilon\nabla-\mathbf{A})^2 u^{\varepsilon}+\frac{V}{i\varepsilon}u^{\varepsilon}\right)\rangle\\
=&\frac{1}{2}\langle\frac{1}{2i\varepsilon}(-i\varepsilon\nabla-\mathbf{A})^2 u^{\varepsilon}+\frac{V}{i\varepsilon}u^{\varepsilon},(-i\varepsilon\nabla-\mathbf{A})^2u^{\varepsilon}\rangle\\
&+\frac{1}{2}\langle(-i\varepsilon\nabla-\mathbf{A})^2 u^{\varepsilon},\frac{1}{2i\varepsilon}(-i\varepsilon\nabla-\mathbf{A})^2 u^{\varepsilon}+\frac{V}{i\varepsilon}u^{\varepsilon}\rangle\\
=&\frac{1}{2i\varepsilon}\langle Vu^{\varepsilon},(-i\varepsilon\nabla-\mathbf{A})^2 u^{\varepsilon}\rangle-\frac{1}{2i\varepsilon}\langle(-i\varepsilon\nabla-\mathbf{A})^2 u^{\varepsilon},Vu^{\varepsilon}\rangle. 
\end{align*}
类似的,
\begin{align*}
(II)=&\langle \frac{1}{2i\varepsilon}(-i\varepsilon\nabla-\mathbf{A})^2 u^{\varepsilon}+\frac{V}{i\varepsilon}u^{\varepsilon},Vu^{\varepsilon}\rangle+\langle Vu^{\varepsilon},\frac{1}{2i\varepsilon}(-i\varepsilon\nabla-\mathbf{A})^2 u^{\varepsilon}
+\frac{V}{i\varepsilon}u^{\varepsilon}\rangle \\
=&\frac{1}{2i\varepsilon}\langle(-i\varepsilon\nabla-\mathbf{A})^2 u^{\varepsilon},Vu^{\varepsilon}\rangle-\frac{1}{2i\varepsilon}\langle Vu^{\varepsilon},(-i\varepsilon\nabla-\mathbf{A})^2 u^{\varepsilon}\rangle = -(I).
\end{align*}
因此我们有$(I)+(II)=0$,也就说明了$\mathcal E(t) = \mathcal E(0)$。
\end{proof}

